\documentclass[aspectratio=169,11pt]{beamer}
\usetheme{metropolis}
\usepackage{graphicx}
\usepackage{amsmath}
\usepackage{booktabs}

% Reduce bottom margin for captions
\setbeamertemplate{frametitle}[default][left]
\addtobeamertemplate{frametitle}{}{\vspace{-0.3em}}

\title{Sensorimotor Habituation in \textit{Drosophila} Larvae}
\subtitle{Population-Level Modeling and Individual Phenotyping Validation}
\author{Gil Raitses}
\institute{Syracuse University}
\date{December 22, 2025}

\begin{document}

\begin{frame}
\titlepage
\end{frame}

%% ===========================================
%% PART 1: ORIGINAL STUDY
%% ===========================================

\begin{frame}{Executive Summary -- Original Study}
\textbf{Population-Level Sensorimotor Habituation Model}

\vspace{0.3cm}
Larval reorientation behavior follows a gamma-difference kernel with two timescales.

\vspace{0.2cm}
The fast excitatory component with $\tau_1 \approx 0.3$ seconds drives the initial response to light onset.

\vspace{0.2cm}
The slow inhibitory component with $\tau_2 \approx 4$ seconds produces delayed suppression.

\vspace{0.2cm}
Model validated across 14 experiments with 701 tracks.

\vspace{0.5cm}
\textbf{Key Result} \quad The gamma-difference kernel accurately predicts population-level reorientation dynamics.
\end{frame}

\begin{frame}{Kernel Structure}
\centering
\includegraphics[height=0.72\textheight]{../../figures/figure1_kernel.png}

\vspace{0.2cm}
\small The gamma-difference kernel combines fast excitation peaking at 0.3 seconds with slow suppression. The left panel shows the combined kernel shape. The right panel shows the fast excitatory component in green and the slow suppressive component in red.
\end{frame}

\begin{frame}{Simulated vs Empirical Event Counts}
\centering
\includegraphics[height=0.72\textheight]{../figures/fig_simulation_design.pdf}

\vspace{0.2cm}
\small Panel A shows the overlapping distributions of event counts for empirical tracks with n=260 and simulated tracks with n=300. Panel B shows the median and interquartile range for both distributions. Simulated tracks match the empirical distribution.
\end{frame}

\begin{frame}{Habituation Dynamics}
\centering
\includegraphics[height=0.68\textheight]{../../figures/figure_habituation.png}

\vspace{0.2cm}
\small Turn fraction increases across pulse number in all four experimental conditions. The slope indicates habituation rate. Higher slopes in cycling conditions with 50-250 PWM suggest stronger habituation. Error bands show 95\% confidence intervals.
\end{frame}

\begin{frame}{Leave-One-Experiment-Out Validation}
\centering
\includegraphics[height=0.68\textheight]{../../figures/loeo_permutation_null.png}

\vspace{0.2cm}
\small LOEO permutation test assesses whether kernel parameters generalize across experiments. The null distribution was generated from 1000 permutations of experiment labels. The observed pass rate of 50\% falls within the null distribution with p=0.618.
\end{frame}

%% ===========================================
%% PART 2: FOLLOW-UP STUDY
%% ===========================================

\begin{frame}{Executive Summary -- Follow-Up Study}
\textbf{Individual-Level Phenotyping Validation}

\vspace{0.3cm}
Can individual larvae be phenotyped using kernel parameters?

\vspace{0.2cm}
Challenge: Sparse data with only 18 to 25 events per 10 to 20 minute track.

\vspace{0.2cm}
Finding: Apparent phenotypic clusters are artifacts of sparse data.

\vspace{0.2cm}
Gap statistic suggests optimal k=1 cluster indicating no discrete phenotypes.

\vspace{0.2cm}
Only 8.6\% of tracks show genuine individual differences.

\vspace{0.4cm}
\textbf{Key Result} \quad Individual-level phenotyping requires protocol modifications.
\end{frame}

%% Clustering Illusion
\begin{frame}{PCA Distribution}
\centering
\includegraphics[height=0.72\textheight]{figures/clustering_panel_A.pdf}

\vspace{0.2cm}
\small Principal component analysis of kernel parameters reveals a unimodal distribution. Points are not colored by cluster assignment because no clustering was applied. The continuous spread indicates no natural groupings exist in the parameter space.
\end{frame}

\begin{frame}{Validation Failures}
\centering
\includegraphics[height=0.72\textheight]{figures/clustering_panel_B.pdf}

\vspace{0.2cm}
\small Four validation methods tested whether clusters are reproducible. All methods failed with ARI scores below 0.13. The green dashed line at 0.5 indicates the success threshold. Round-trip validation simulates from fitted parameters and re-clusters.
\end{frame}

\begin{frame}{Gap Statistic}
\centering
\includegraphics[height=0.72\textheight]{figures/clustering_panel_C.pdf}

\vspace{0.2cm}
\small The gap statistic compares within-cluster dispersion to that expected under a null reference distribution. The minimum at k=1 indicates that a single cluster best describes the data. There is no evidence for discrete phenotypic groups.
\end{frame}

%% Data Sparsity
\begin{frame}{Event Distribution}
\centering
\includegraphics[height=0.72\textheight]{figures/sparsity_panel_A.pdf}

\vspace{0.2cm}
\small Histogram of reorientation events per track. The red line shows the observed mean of 31 events. The green dashed line shows the recommended minimum of 100 events for stable 6-parameter estimation. Current data falls far short of this requirement.
\end{frame}

\begin{frame}{MLE Estimate Instability}
\centering
\includegraphics[height=0.72\textheight]{figures/sparsity_panel_B.pdf}

\vspace{0.2cm}
\small Maximum likelihood estimates for $\tau_1$ span 0 to 5 seconds. Green bars indicate biologically plausible values between 0.3 and 1.5 seconds. Red bars indicate fitting failures. Many tracks produce implausible estimates due to sparse data.
\end{frame}

\begin{frame}{Shrinkage Effect}
\centering
\includegraphics[height=0.72\textheight]{figures/sparsity_panel_C.pdf}

\vspace{0.2cm}
\small Hierarchical Bayesian estimates versus MLE estimates. The dashed diagonal indicates no shrinkage. The purple horizontal line indicates the population mean at 0.63 seconds. Bayesian estimates shrink toward the population mean.
\end{frame}

%% Identifiability
\begin{frame}{Same Events -- Different Information}
\centering
\includegraphics[height=0.72\textheight]{figures/identifiability_panel_A.pdf}

\vspace{0.2cm}
\small Both designs yield approximately 17 events per track. Continuous design produces high bias at 0.61 seconds and high RMSE at 0.71 seconds. Burst design produces low bias at 0.14 seconds and low RMSE at 0.38 seconds.
\end{frame}

\begin{frame}{Fisher Information}
\centering
\includegraphics[height=0.72\textheight]{figures/identifiability_panel_B.pdf}

\vspace{0.2cm}
\small Fisher Information quantifies how much information each event contains about $\tau_1$. Burst design extracts 10 times more information per event than continuous design.
\end{frame}

%% Stimulation Protocols
\begin{frame}{Current Protocol -- Continuous 10s}
\centering
\includegraphics[width=0.92\textwidth]{figures/stimulation_panel_A.pdf}
\end{frame}

\begin{frame}{Recommended Protocol -- Burst 10x0.5s}
\centering
\includegraphics[width=0.92\textwidth]{figures/stimulation_panel_B.pdf}
\end{frame}

%% Kernel Comparison
\begin{frame}{Kernel Model Comparison}
\centering
\includegraphics[height=0.68\textheight]{figures/fig_kernel_comparison.pdf}

\vspace{0.2cm}
\small Both kernels were fitted to the same empirical PSTH. The gamma-difference kernel achieves R$^2$ = 0.968 with 6 parameters. The raised cosine basis achieves R$^2$ = 0.974 with 12 parameters. The simpler model captures nearly identical fit quality.
\end{frame}

%% ===========================================
%% RECOMMENDATIONS - 5 SLIDES WITH FIGURES
%% ===========================================

\begin{frame}{Recommendation 1 -- Protocol Modification}
\centering
\includegraphics[width=0.92\textwidth]{figures/stimulation_panel_B.pdf}

\vspace{0.2cm}
\small Replace continuous 10 second ON periods with burst trains. Use 10 pulses of 0.5 seconds separated by 0.5 second gaps. Burst design samples the early excitatory window repeatedly and achieves 8 times more informative events.
\end{frame}

\begin{frame}{Recommendation 2 -- Extended Recording}
\centering
\includegraphics[height=0.68\textheight]{figures/sparsity_panel_A.pdf}

\vspace{0.2cm}
\small Target 40 or more minutes of recording to achieve at least 50 events per track. Current 10 to 20 minute recordings yield only 18 to 25 events. The data-to-parameter ratio should exceed 10 to 1 for stable estimation.
\end{frame}

\begin{frame}{Recommendation 3 -- Model Simplification}
\centering
\includegraphics[height=0.68\textheight]{figures/sparsity_panel_C.pdf}

\vspace{0.2cm}
\small Fix $\tau_2$ and amplitude parameters A and B at population values. Estimate only $\tau_1$ per individual. This reduces the parameter space from 6 dimensions to 1 dimension and dramatically improves identifiability.
\end{frame}

\begin{frame}{Recommendation 4 -- Alternative Phenotypes}
\centering
\includegraphics[height=0.68\textheight]{figures/identifiability_panel_A.pdf}

\vspace{0.2cm}
\small Use composite phenotypes that are robust with sparse data. The ON/OFF event ratio requires only event counts not kernel fitting. First-event latency measures response speed directly. Both avoid the identifiability problem entirely.
\end{frame}

\begin{frame}{Recommendation 5 -- Within-Condition Analysis}
\centering
\includegraphics[height=0.68\textheight]{figures/clustering_panel_B.pdf}

\vspace{0.2cm}
\small Analyze individual differences within experimental conditions not across them. Condition effects dominate individual effects when data is pooled. The ARI near zero indicates no reproducible structure across methods.
\end{frame}

%% ===========================================
%% CONCLUSIONS
%% ===========================================

\begin{frame}{Conclusions -- Original Study}
\textbf{Population-Level Modeling Success}

\vspace{0.4cm}
The gamma-difference kernel accurately models population-level reorientation dynamics.

\vspace{0.3cm}
Two timescales govern behavior. Fast excitation with $\tau_1 \approx 0.3$ seconds captures initial sensory response. Slow suppression with $\tau_2 \approx 4$ seconds captures habituation.

\vspace{0.3cm}
Model is robust across 14 experiments via LOEO cross-validation.
\end{frame}

\begin{frame}{Conclusions -- Follow-Up Study}
\textbf{Individual Phenotyping Challenges}

\vspace{0.4cm}
Individual phenotyping fails with current protocols due to sparse data.

\vspace{0.3cm}
Apparent clusters are statistical artifacts rather than genuine phenotypes.

\vspace{0.3cm}
Only 8.6\% of tracks show genuine individual differences.

\vspace{0.3cm}
Current protocols achieve only 20 to 30\% power for phenotype detection.

\vspace{0.5cm}
\textbf{Bottom Line} \quad Population-level analysis is robust. Individual phenotyping requires experimental redesign.
\end{frame}

\begin{frame}
\centering
\Huge Thank You

\vspace{1cm}
\Large Questions?
\end{frame}

\end{document}

