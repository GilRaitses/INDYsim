\documentclass[aspectratio=169,11pt]{beamer}
\usetheme{metropolis}
\usepackage{graphicx}
\usepackage{amsmath}
\usepackage{booktabs}

\title{Sensorimotor Habituation in \textit{Drosophila} Larvae}
\subtitle{Population-Level Modeling and Individual Phenotyping Validation}
\author{Gil Raitses}
\institute{Syracuse University}
\date{December 22, 2025}

\begin{document}

\begin{frame}
\titlepage
\end{frame}

%% ===========================================
%% PART 1: ORIGINAL STUDY
%% ===========================================

\begin{frame}{Executive Summary -- Original Study}
\textbf{Population-Level Sensorimotor Habituation Model}

\begin{itemize}
    \item Larval reorientation behavior follows a \textbf{gamma-difference kernel} with two timescales
    \item Fast excitatory component with $\tau_1 \approx 0.3$ seconds drives the initial response
    \item Slow inhibitory component with $\tau_2 \approx 4$ seconds produces delayed suppression
    \item Model validated across 14 experiments with 701 tracks
\end{itemize}

\vspace{0.5cm}
\textbf{Key Result} \quad The gamma-difference kernel accurately predicts population-level reorientation dynamics.
\end{frame}

\begin{frame}{Kernel Structure}
\centering
\includegraphics[height=0.8\textheight]{../../figures/figure1_kernel.png}
\end{frame}

\begin{frame}{Habituation Dynamics}
\centering
\includegraphics[height=0.8\textheight]{../../figures/figure_habituation.png}
\end{frame}

\begin{frame}{Leave-One-Experiment-Out Validation}
\centering
\includegraphics[height=0.8\textheight]{../../figures/loeo_permutation_null.png}
\end{frame}

%% ===========================================
%% PART 2: FOLLOW-UP STUDY
%% ===========================================

\begin{frame}{Executive Summary -- Follow-Up Study}
\textbf{Individual-Level Phenotyping Validation}

\begin{itemize}
    \item \textbf{Question} \quad Can individual larvae be phenotyped using kernel parameters?
    \item \textbf{Challenge} \quad Sparse data with only 18 to 25 events per track
    \item \textbf{Finding} \quad Apparent phenotypic clusters are artifacts of sparse data
    \item Only 8.6\% of tracks show genuine individual differences
\end{itemize}

\vspace{0.5cm}
\textbf{Key Result} \quad Individual-level phenotyping requires protocol modifications.
\end{frame}

%% Clustering Illusion - CLEAN PANELS
\begin{frame}{PCA Distribution}
\centering
\includegraphics[height=0.85\textheight]{figures/clustering_panel_A.pdf}
\end{frame}

\begin{frame}{Validation Failures}
\centering
\includegraphics[height=0.85\textheight]{figures/clustering_panel_B.pdf}
\end{frame}

\begin{frame}{Gap Statistic}
\centering
\includegraphics[height=0.85\textheight]{figures/clustering_panel_C.pdf}
\end{frame}

%% Data Sparsity - CLEAN PANELS
\begin{frame}{Event Distribution}
\centering
\includegraphics[height=0.85\textheight]{figures/sparsity_panel_A.pdf}
\end{frame}

\begin{frame}{MLE Estimate Instability}
\centering
\includegraphics[height=0.85\textheight]{figures/sparsity_panel_B.pdf}
\end{frame}

\begin{frame}{Shrinkage Effect}
\centering
\includegraphics[height=0.85\textheight]{figures/sparsity_panel_C.pdf}
\end{frame}

\begin{frame}{More Data Reduces Uncertainty}
\centering
\includegraphics[height=0.85\textheight]{figures/sparsity_panel_D.pdf}
\end{frame}

%% Identifiability - CLEAN PANELS
\begin{frame}{Same Events, Different Information}
\centering
\includegraphics[height=0.85\textheight]{figures/identifiability_panel_A.pdf}
\end{frame}

\begin{frame}{Fisher Information Comparison}
\centering
\includegraphics[height=0.85\textheight]{figures/identifiability_panel_B.pdf}
\end{frame}

\begin{frame}{MLE Recovery by Design}
\centering
\includegraphics[height=0.85\textheight]{figures/identifiability_panel_C.pdf}
\end{frame}

\begin{frame}{Why Continuous Design Fails}
\centering
\includegraphics[height=0.85\textheight]{figures/identifiability_panel_D.pdf}
\end{frame}

%% Stimulation Protocols - CLEAN PANELS
\begin{frame}{Current Protocol -- Continuous 10s}
\centering
\includegraphics[width=0.95\textwidth]{figures/stimulation_panel_A.pdf}
\end{frame}

\begin{frame}{Recommended Protocol -- Burst 10x0.5s}
\centering
\includegraphics[width=0.95\textwidth]{figures/stimulation_panel_B.pdf}
\end{frame}

\begin{frame}{Alternative -- Medium 4x1s}
\centering
\includegraphics[width=0.95\textwidth]{figures/stimulation_panel_C.pdf}
\end{frame}

\begin{frame}{Alternative -- Long 2x2s}
\centering
\includegraphics[width=0.95\textwidth]{figures/stimulation_panel_D.pdf}
\end{frame}

%% Kernel Comparison
\begin{frame}{Kernel Model Comparison}
\centering
\includegraphics[height=0.8\textheight]{../figures/fig_kernel_comparison.pdf}

The gamma-difference kernel with 6 parameters achieves $R^2 = 0.968$ compared to the raised cosine basis with 12 parameters.
\end{frame}

%% ===========================================
%% CONCLUSIONS
%% ===========================================

\begin{frame}{Conclusions -- Original Study}
\textbf{Population-Level Modeling Success}

\begin{itemize}
    \item Gamma-difference kernel accurately models population-level reorientation dynamics
    \item Two timescales govern behavior
    \begin{itemize}
        \item Fast excitation $\tau_1 \approx 0.3$ seconds for initial sensory response
        \item Slow suppression $\tau_2 \approx 4$ seconds for habituation
    \end{itemize}
    \item Robust across 14 experiments via LOEO cross-validation
\end{itemize}
\end{frame}

\begin{frame}{Conclusions -- Follow-Up Study}
\textbf{Individual Phenotyping Challenges}

\begin{itemize}
    \item Individual phenotyping fails with current protocols due to sparse data
    \item Apparent clusters are statistical artifacts rather than genuine phenotypes
    \item Only 8.6\% of tracks show genuine individual differences
    \item Current protocols achieve only 20 to 30\% power for phenotype detection
\end{itemize}

\vspace{0.5cm}
\textbf{Bottom Line} \quad Population-level analysis is robust. Individual phenotyping requires experimental redesign.
\end{frame}

\begin{frame}{Recommendations}
\begin{enumerate}
    \item \textbf{Protocol modification} \quad Replace continuous 10s ON with burst trains using 10 pulses of 0.5s each
    \item \textbf{Extended recording} \quad Target 40 or more minutes to achieve at least 50 events per track
    \item \textbf{Model simplification} \quad Fix $\tau_2$, $A$, and $B$ at population values then estimate only $\tau_1$
    \item \textbf{Alternative phenotypes} \quad Use ON/OFF ratio and first-event latency which are robust with sparse data
\end{enumerate}
\end{frame}

\begin{frame}
\centering
\Huge Thank You

\vspace{1cm}
\Large Questions?
\end{frame}

\end{document}

