\documentclass[aspectratio=169,11pt]{beamer}
\usetheme{metropolis}
\usepackage{graphicx}
\usepackage{amsmath}
\usepackage{booktabs}

% Colors
\definecolor{darkblue}{RGB}{0,51,102}
\definecolor{lightgray}{RGB}{245,245,245}

\title{Sensorimotor Habituation in \textit{Drosophila} Larvae}
\subtitle{Population-Level Modeling and Individual Phenotyping Validation}
\author{Gil Raitses}
\institute{Syracuse University}
\date{December 22, 2025}

\begin{document}

\begin{frame}
\titlepage
\end{frame}

%% ===========================================
%% PART 1: ORIGINAL STUDY
%% ===========================================

\begin{frame}{Executive Summary -- Original Study}
\textbf{Population-Level Sensorimotor Habituation Model}

\begin{itemize}
    \item Larval reorientation behavior follows a \textbf{gamma-difference kernel} with two timescales
    \item Fast excitatory component with $\tau_1 \approx 0.3$ seconds drives the initial response to light onset
    \item Slow inhibitory component with $\tau_2 \approx 4$ seconds produces delayed suppression
    \item Model validated across 14 experiments with 701 tracks
    \item Leave-one-experiment-out cross-validation confirms robustness
\end{itemize}

\vspace{0.5cm}
\textbf{Key Result} \quad The gamma-difference kernel accurately predicts population-level reorientation dynamics under optogenetic stimulation.
\end{frame}

\begin{frame}{Kernel Structure}
\centering
\includegraphics[height=0.75\textheight]{../../figures/figure1_kernel.png}

\vspace{0.3cm}
The gamma-difference kernel combines fast excitation peaking at 0.3 seconds with slow suppression persisting for 4 seconds. The kernel modulates the baseline hazard rate of reorientation events.
\end{frame}

%% Model Validation - Split into panels
\begin{frame}{Model Validation -- Hazard Profile}
\centering
\includegraphics[height=0.75\textheight,clip,trim=0 200 400 0]{../../figures/figure2_validation.png}

\vspace{0.3cm}
\textbf{Panel A} \quad The hazard rate profile shows elevated reorientation probability during LED-ON periods. The kernel accurately captures the temporal dynamics of event probability.
\end{frame}

\begin{frame}{Model Validation -- LED Phase Event Rates}
\centering
\includegraphics[height=0.75\textheight,clip,trim=400 200 0 0]{../../figures/figure2_validation.png}

\vspace{0.3cm}
\textbf{Panel B} \quad Event rates during LED-ON versus LED-OFF phases. Empirical data shows 2$\times$ suppression ratio. Simulated events match the empirical distribution.
\end{frame}

%% Trajectory Analysis - Keep as overview
\begin{frame}{Trajectory Analysis -- Three Example Tracks}
\centering
\includegraphics[height=0.75\textheight]{../../figures/figure3_trajectories.png}

\vspace{0.3cm}
Three simulated larval trajectories showing reorientation events as red dots aligned to LED stimulation cycles shown in yellow. Event clustering occurs after each LED onset.
\end{frame}

%% Habituation
\begin{frame}{Habituation Dynamics}
\centering
\includegraphics[height=0.75\textheight]{../../figures/figure_habituation.png}

\vspace{0.3cm}
Response magnitude decreases across repeated stimulation cycles. Four experimental conditions show consistent habituation slopes ranging from 0.015 to 0.038 per pulse.
\end{frame}

%% Factorial Design - Split
\begin{frame}{Factorial Design -- Suppression Amplitude}
\centering
\includegraphics[height=0.75\textheight,clip,trim=0 0 350 0]{../../figures/figure5_factorial.png}

\vspace{0.3cm}
\textbf{Panel A} \quad Suppression amplitude varies with LED intensity stepping. Higher intensity steps produce stronger suppression. Values show normalized amplitude.
\end{frame}

\begin{frame}{Factorial Design -- Effect Estimates}
\centering
\includegraphics[height=0.75\textheight,clip,trim=350 0 0 0]{../../figures/figure5_factorial.png}

\vspace{0.3cm}
\textbf{Panel B} \quad Factorial effect estimates with 95\% confidence intervals. Stars indicate significant effects. Cycling and intensity modulate suppression parameters.
\end{frame}

%% Behavioral State - Keep as overview (4 conditions)
\begin{frame}{Behavioral State Analysis}
\centering
\includegraphics[height=0.75\textheight]{../../figures/figure_behavior_stacked_cinnamoroll.png}

\vspace{0.3cm}
Fractional time in behavioral states across four experimental conditions. Turn fraction increases during LED-ON periods across all conditions.
\end{frame}

%% LOEO Validation
\begin{frame}{Leave-One-Experiment-Out Validation}
\centering
\includegraphics[height=0.75\textheight]{../../figures/loeo_permutation_null.png}

\vspace{0.3cm}
LOEO permutation test with n=1000 permutations. Observed pass rate of 50\% matches null distribution mean of 50.5\%. The p-value of 0.618 indicates the kernel generalizes across experiments.
\end{frame}

%% ===========================================
%% PART 2: FOLLOW-UP STUDY
%% ===========================================

\begin{frame}{Executive Summary -- Follow-Up Study}
\textbf{Individual-Level Phenotyping Validation}

\begin{itemize}
    \item \textbf{Question} \quad Can individual larvae be phenotyped using kernel parameters?
    \item \textbf{Challenge} \quad Sparse data with only 18 to 25 events per 10 to 20 minute track
    \item \textbf{Finding} \quad Apparent phenotypic clusters are artifacts of sparse data
    \item Gap statistic suggests optimal k=1 cluster indicating no discrete phenotypes
    \item Round-trip validation achieves ARI = 0.128 which falls below the 0.5 threshold
    \item Only 8.6\% of tracks show genuine individual differences
\end{itemize}

\vspace{0.3cm}
\textbf{Key Result} \quad Population-level analysis is robust. Individual-level phenotyping requires protocol modifications.
\end{frame}

%% Clustering Illusion - Split into 3 slides
\begin{frame}{The Clustering Illusion -- PCA Distribution}
\centering
\includegraphics[height=0.75\textheight,clip,trim=0 0 500 0]{../figures/core/fig1_clustering_illusion.pdf}

\vspace{0.3cm}
\textbf{Panel A} \quad Principal component analysis of kernel parameters reveals a unimodal distribution rather than discrete clusters. No natural groupings emerge from the data.
\end{frame}

\begin{frame}{The Clustering Illusion -- Validation Failures}
\centering
\includegraphics[height=0.75\textheight,clip,trim=250 0 250 0]{../figures/core/fig1_clustering_illusion.pdf}

\vspace{0.3cm}
\textbf{Panel B} \quad All four validation methods failed with ARI scores below 0.13. Round-trip ARI of 0.128 indicates clusters are not reproducible. PSTH-kernel and FNO-parametric comparisons show near-zero agreement.
\end{frame}

\begin{frame}{The Clustering Illusion -- Gap Statistic}
\centering
\includegraphics[height=0.75\textheight,clip,trim=500 0 0 0]{../figures/core/fig1_clustering_illusion.pdf}

\vspace{0.3cm}
\textbf{Panel C} \quad Gap statistic analysis. The optimal number of clusters is k=1. No evidence exists for discrete phenotypic groups in the data.
\end{frame}

%% Data Sparsity - Split into 2-3 slides
\begin{frame}{Data Sparsity Challenge -- Event Count Distribution}
\centering
\includegraphics[height=0.75\textheight,clip,trim=0 0 500 0]{../figures/core/fig2_data_sparsity.pdf}

\vspace{0.3cm}
\textbf{Panel A} \quad Tracks contain only 25 events on average. The recommended minimum is 100 events for stable 6-parameter estimation. Current data-to-parameter ratio is 3 to 1 instead of 10 to 1.
\end{frame}

\begin{frame}{Data Sparsity Challenge -- MLE Instability}
\centering
\includegraphics[height=0.75\textheight,clip,trim=250 0 250 0]{../figures/core/fig2_data_sparsity.pdf}

\vspace{0.3cm}
\textbf{Panel B} \quad Maximum likelihood estimates for $\tau_1$ span 0 to 5 seconds. The biological range is 0.3 to 1.5 seconds. Many estimates fall outside this range indicating fitting failures.
\end{frame}

%% Hierarchical Shrinkage - Split
\begin{frame}{Hierarchical Shrinkage -- Population Overlap}
\centering
\includegraphics[height=0.75\textheight,clip,trim=0 0 400 0]{../figures/core/fig3_hierarchical_shrinkage.pdf}

\vspace{0.3cm}
\textbf{Panel A} \quad The hierarchical Bayesian model reveals 91\% of tracks overlap with the population mean. Only 22 out of 256 tracks representing 8.6\% show genuine individual differences.
\end{frame}

\begin{frame}{Hierarchical Shrinkage -- Shrinkage Effect}
\centering
\includegraphics[height=0.75\textheight,clip,trim=400 0 0 0]{../figures/core/fig3_hierarchical_shrinkage.pdf}

\vspace{0.3cm}
\textbf{Panel B} \quad Bayesian estimates shrink toward the population mean of $\tau_1 = 0.63$ seconds. Extreme MLE values are pulled back. Only outliers in magenta remain distinct from the population.
\end{frame}

%% Fast Responders - Split
\begin{frame}{Candidate Fast Responders -- Violin Comparison}
\centering
\includegraphics[height=0.75\textheight,clip,trim=0 0 400 0]{../figures/core/fig4_fast_responders.pdf}

\vspace{0.3cm}
\textbf{Panel A} \quad Violin plots comparing normal tracks with n=234 to outliers with n=22. Outliers show $\tau_1 \approx 0.45$ seconds versus population mean of 0.63 seconds. The difference requires independent validation.
\end{frame}

\begin{frame}{Candidate Fast Responders -- Kernel Shape Difference}
\centering
\includegraphics[height=0.75\textheight,clip,trim=400 0 0 0]{../figures/core/fig4_fast_responders.pdf}

\vspace{0.3cm}
\textbf{Panel B} \quad Kernel shape comparison. Fast responders shown in orange peak earlier than the population kernel shown in blue. The difference in peak timing is approximately 0.2 seconds.
\end{frame}

%% Power Analysis
\begin{frame}{Power Analysis}
\centering
\includegraphics[height=0.75\textheight]{../figures/fig5_power_analysis.pdf}

\vspace{0.3cm}
Current data achieves only 20 to 30\% power to detect a $\tau_1$ difference of 0.2 seconds. Reaching 80\% power requires approximately 100 to 120 events per track.
\end{frame}

%% Identifiability - Split into 2 slides
\begin{frame}{Identifiability Problem -- Information by Design}
\centering
\includegraphics[height=0.75\textheight,clip,trim=0 0 400 0]{../figures/fig2_identifiability_v3.pdf}

\vspace{0.3cm}
\textbf{Panels A and B} \quad Burst stimulation provides 10$\times$ higher Fisher Information for $\tau_1$ than continuous stimulation. The same number of events yields dramatically different estimation precision.
\end{frame}

\begin{frame}{Identifiability Problem -- Why Continuous Design Fails}
\centering
\includegraphics[height=0.75\textheight,clip,trim=400 0 0 0]{../figures/fig2_identifiability_v3.pdf}

\vspace{0.3cm}
\textbf{Panels C and D} \quad With inhibition-dominated kernels where B/A exceeds 8, 80\% of events occur during LED-OFF periods. These events contain no information about $\tau_1$. Burst design samples the early excitatory window.
\end{frame}

%% Design Comparison - Split into 2 slides
\begin{frame}{Design Comparison -- Event Yield and Error}
\centering
\includegraphics[height=0.75\textheight,clip,trim=0 0 350 0]{../figures/fig_design_comparison_summary.pdf}

\vspace{0.3cm}
\textbf{Panels A and B} \quad Burst design yields 8$\times$ more events per track than continuous design. Estimation error RMSE decreases from 0.71 seconds to 0.38 seconds with burst stimulation.
\end{frame}

\begin{frame}{Design Comparison -- Recommendations}
\centering
\includegraphics[height=0.75\textheight,clip,trim=350 0 0 0]{../figures/fig_design_comparison_summary.pdf}

\vspace{0.3cm}
\textbf{Panels C and D} \quad Recommended protocol uses burst stimulation with 10 pulses of 0.5 seconds ON with 0.5 second gaps. Alternative strategies include composite phenotypes and first-event latency.
\end{frame}

%% Stimulation Protocol - Split into 2 slides
\begin{frame}{Stimulation Protocols -- Current vs Recommended}
\centering
\includegraphics[height=0.75\textheight,clip,trim=0 150 0 200]{../figures/fig_stimulation_schematic.pdf}

\vspace{0.3cm}
\textbf{Panels A and B} \quad Current design uses 10 seconds ON followed by 20 seconds OFF. Recommended burst design uses 10 pulses of 0.5 seconds separated by 0.5 second gaps achieving 8$\times$ more informative events.
\end{frame}

\begin{frame}{Stimulation Protocols -- Alternative Designs}
\centering
\includegraphics[height=0.75\textheight,clip,trim=0 0 0 300]{../figures/fig_stimulation_schematic.pdf}

\vspace{0.3cm}
\textbf{Panels C and D} \quad Medium design with 4 pulses of 1 second and Long design with 2 pulses of 2 seconds. Event yield varies but all alternatives outperform continuous stimulation for $\tau_1$ estimation.
\end{frame}

%% Kernel Model Comparison
\begin{frame}{Kernel Model Comparison}
\centering
\includegraphics[height=0.75\textheight]{../figures/fig_kernel_comparison.pdf}

\vspace{0.3cm}
The gamma-difference kernel with 6 parameters achieves $R^2 = 0.968$ compared to the raised cosine basis with 12 parameters. The simpler model captures nearly identical dynamics while enabling biological interpretation.
\end{frame}

%% ===========================================
%% CONCLUSIONS
%% ===========================================

\begin{frame}{Conclusions -- Original Study}
\textbf{Population-Level Modeling Success}

\begin{itemize}
    \item Gamma-difference kernel accurately models population-level reorientation dynamics
    \item Two timescales govern behavior
    \begin{itemize}
        \item Fast excitation $\tau_1 \approx 0.3$ seconds for initial sensory response
        \item Slow suppression $\tau_2 \approx 4$ seconds for habituation
    \end{itemize}
    \item Robust across 14 experiments via LOEO cross-validation
    \item Factorial design reveals condition-specific parameter variation
\end{itemize}
\end{frame}

\begin{frame}{Conclusions -- Follow-Up Study}
\textbf{Individual Phenotyping Challenges}

\begin{itemize}
    \item Individual phenotyping fails with current protocols due to sparse data
    \item Apparent clusters are statistical artifacts rather than genuine phenotypes
    \item Only 8.6\% of tracks show genuine individual differences
    \item Current protocols achieve only 20 to 30\% power for phenotype detection
\end{itemize}

\vspace{0.5cm}
\textbf{Bottom Line} \quad Population-level analysis is robust and publishable. Individual phenotyping requires experimental redesign.
\end{frame}

\begin{frame}{Recommendations for Future Work}
\begin{enumerate}
    \item \textbf{Protocol modification} \quad Replace 10 second continuous ON with burst trains using 10 pulses of 0.5 seconds each
    \item \textbf{Extended recording} \quad Target 40 or more minutes to achieve at least 50 events per track
    \item \textbf{Model simplification} \quad Fix $\tau_2$, $A$, and $B$ at population values then estimate only $\tau_1$
    \item \textbf{Alternative phenotypes} \quad Use ON/OFF ratio and first-event latency which are robust with sparse data
    \item \textbf{Within-condition analysis} \quad Avoid confounding by experimental condition effects
\end{enumerate}
\end{frame}

\begin{frame}
\centering
\Huge Thank You

\vspace{1cm}
\Large Questions?
\end{frame}

\end{document}

