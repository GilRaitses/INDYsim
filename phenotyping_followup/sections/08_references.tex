\section{References}

\begin{thebibliography}{9}

\bibitem{tibshirani2001estimating}
Tibshirani, R., Walther, G., \& Hastie, T. (2001).
Estimating the number of clusters in a data set via the gap statistic.
\textit{Journal of the Royal Statistical Society: Series B (Statistical Methodology)}, 63(2), 411-423.

\bibitem{hubert1985comparing}
Hubert, L., \& Arabie, P. (1985).
Comparing partitions.
\textit{Journal of Classification}, 2(1), 193-218.

\bibitem{rousseeuw1987silhouettes}
Rousseeuw, P. J. (1987).
Silhouettes: A graphical aid to the interpretation and validation of cluster analysis.
\textit{Journal of Computational and Applied Mathematics}, 20, 53-65.

\bibitem{pulver2018bioluminescence}
Pulver, S. R., Bayraktar, E., Petrossian, B., \& Kaiser, M. (2018).
Monitoring brain activity and behavior in freely behaving Drosophila larvae using bioluminescence.
\textit{Scientific Reports}, 8(1), 10410.

\bibitem{szuperak2018larval}
Szuperak, M., Churgin, M. A., Borja, A. J., Raizen, D. M., Bhatt, A. S., Kayser, M. S., \& Bhatt, P. J. (2018).
A sleep state in Drosophila larvae required for neural stem cell proliferation.
\textit{eLife}, 7, e33220.

\bibitem{gelman2006data}
Gelman, A., \& Hill, J. (2006).
\textit{Data Analysis Using Regression and Multilevel/Hierarchical Models}.
Cambridge University Press.

\bibitem{betancourt2017hierarchical}
Betancourt, M. (2017).
A conceptual introduction to Hamiltonian Monte Carlo.
\textit{arXiv preprint arXiv:1701.02434}.

\bibitem{shimazaki2007method}
Shimazaki, H., \& Shinomoto, S. (2007).
A method for selecting the bin size of a time histogram.
\textit{Neural Computation}, 19(6), 1503-1527.

\bibitem{gerstein1960some}
Gerstein, G. L., \& Kiang, N. Y. (1960).
An approach to the quantitative analysis of electrophysiological data from single neurons.
\textit{Biophysical Journal}, 1(1), 15-28.

\bibitem{perkel1967neuronal}
Perkel, D. H., Gerstein, G. L., \& Moore, G. P. (1967).
Neuronal spike trains and stochastic point processes: I. The single spike train.
\textit{Biophysical Journal}, 7(4), 391-418.

\bibitem{daley2003introduction}
Daley, D. J., \& Vere-Jones, D. (2003).
\textit{An Introduction to the Theory of Point Processes, Volume I: Elementary Theory and Methods}.
Springer.

\bibitem{heckman1984identifiability}
Heckman, J., \& Singer, B. (1984).
The identifiability of the proportional hazard model.
\textit{The Review of Economic Studies}, 51(2), 231-241.

\bibitem{rebora2014bshazard}
Rebora, P., Salim, A., \& Reilly, M. (2014).
bshazard: A flexible tool for nonparametric smoothing of the hazard function.
\textit{The R Journal}, 6(2), 114-122.

\bibitem{salehi2019learning}
Salehi, F., Trouleau, W., Grossglauser, M., \& Thiran, P. (2019).
Learning Hawkes processes from a handful of events.
\textit{Advances in Neural Information Processing Systems}, 32.

\bibitem{du2016recurrent}
Du, N., Dai, H., Trivedi, R., Upadhyay, U., Gomez-Rodriguez, M., \& Song, L. (2016).
Recurrent marked temporal point processes: Embedding event history to vector.
\textit{Proceedings of the 22nd ACM SIGKDD International Conference on Knowledge Discovery and Data Mining}, 1555-1564.

\bibitem{wang1996hazard}
Wang, J.-L., M\"{u}ller, H.-G., \& Eubank, R. L. (1996).
Hazard rate regression using ordinary nonparametric regression smoothers.
\textit{Journal of Computational and Graphical Statistics}, 5(3), 195-212.

\end{thebibliography}


